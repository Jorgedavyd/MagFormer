\documentclass[12pt]{article}

\usepackage{amsmath}
\usepackage{amsfonts}
\usepackage{graphicx}
\usepackage{hyperref}
\usepackage{cite}

\title{Physics informed Multi-Modal Networks for Solar Wind modeling}
\author{Jorge Enciso}
\date{\today}

\begin{document}

\maketitle

\begin{abstract}
    %%TODO
\end{abstract}

\tableofcontents
\newpage

\section{Introduction}

The solar wind phenomena is a vastly known event that arises from ionized solar outbursts into the stellar medium, commonly stimulating planets' magnetosphere. \cite{Gosling2007} The first record of a related event dates from 1859, popularly named "The Carrington event" after the English astronomer Richard Carrington (1826, 1875), who settled the intuition upon solar flares and geomagnetic fluctuations. Individuals from all around the glove sighted the Northern Lights, subtle variations in atmosphere's hues' tones, caused by ionized particles flowing through the ionosphere.

Several attempts to model Solar Wind's dynamics can be pinpointed empirically: analytical methods \cite{}, numerical modelling \cite{2006LNCS.3732..554E, 10.3389/fspas.2023.1105797, windmodelling1}, and machine learning approaches \cite{comp_1, comp_2, comp_3}. All of them with different purposes.

Traditional models and methodologies fail to account for the anysotropic nature of the Solar Wind. The Magnetohydrodynamics Model assumes a Boltzmann-Maxwellian distribution, doted with standard thermodynamical behavior. On the other hand, the Solar Wind is a collisionles plasma, precluding its inclusion solely within those boundaries. Thus, the present work emphasizes the usage of learnable neural operators engineered to analytically and empirically fit kinetic descriptions.

Moreover, kinetic frames based on Maxwellian electromagnetism tend to be computationally expensive for the microscopic and macroscopic frame:

\begin{equation}
    n_{\alpha} (\mathbf{r}, t) := \int f_{\alpha}(\mathbf{r}, \mathbf{v}, t) d^3v
\end{equation}

\begin{equation}
    \tau(r, t) := \sum q_{\alpha} n_{\alpha}
\end{equation}

\begin{equation}
    \mathbf{u_{\alpha}} (\mathbf{r}, t) := \frac{1}{n_{\alpha}(\mathbf{r}, t)} \int v f_{\alpha}(\mathbf{r}, v, t) d^3v
\end{equation}

\begin{equation}
    j(\mathbf{r}, t) := \sum q_{\alpha} n_{\alpha} \mathbf{u_{\alpha}}
\end{equation}

where $n_{\alpha}$ is the particle density, $u_{\alpha}$ the particle's drift velocity, $\tau(r, t)$ is the charge density, $j(r, t)$ is the current density, and $f_{\alpha}(r, v, t)$ is the Boltzmann's density function.

Yet modern physics informed machine learning research schemes enable faster inference time and less computational load: embedding systems' equations within the loss function and constraining network's backbone based on empirical approaches.

\section{Related Work}

\subsection{Solar Wind modelling}

\subsection{Physics informed Machine Learning}

Magnetohydrodynamics (MHD) is the study of the behavior of electrically conducting fluids in the presence of magnetic fields. The governing equations of MHD consist of the magnetohydrodynamic induction equation and the momentum equation, which couple the dynamics of the fluid and magnetic fields. These systems of partial differential equations are complex and non-linear, making them challenging to solve with traditional numerical methods.

Physics-Informed Neural Networks (PINNs) offer a powerful framework for solving PDEs by embedding the physical laws directly into the loss function during the network's training. This approach has shown promising results for solving a variety of PDEs in physics and engineering. In this work, we investigate the use of PINNs to solve MHD PDEs and propose a custom kernel that incorporates the physical terms of MHD, such as advection, induction, and Lorentz forces, into the neural network's architecture.

\section{Data}

\subsection{DSCOVR: Deep Space Climate Observatory}
DSCOVR, a joint mission between NASA and the National Oceanic and Atmospheric Administration, is a crucial observational platform for monitoring space weather \cite{nasa_dscovr}. Launched in 2015, DSCOVR's primary mission is to monitor and provide advanced warning of potentially hazardous space weather events such as solar flares and coronal mass ejections that could impact Earth.

It is equipped with two key instruments for measuring both energetic particle incidence and magnetic field parameters: the Faraday cup and the magnetometer from the PlasMag instrument \cite{nasa_dscovr}. The readings from these two sensors are crucial for virtually analyzing plasma dynamics near the L1 Lagrange point. These readings will be used as part of the core model data due to their real-time availability.

\subsection{ACE: Advanced Composition Explorer}
ACE, launched in 1997, provides continuous measurements of the solar wind and interstellar particles. It is equipped with several instruments designed to study the composition of solar and galactic particles, which are crucial for understanding the space weather environment. ACE's data helps in predicting geomagnetic storms and contributes to our understanding of the heliosphere.

\subsection{WIND}
The WIND spacecraft, launched in 1994, is part of the Global Geospace Science initiative. It provides comprehensive measurements of the solar wind, magnetic fields, and energetic particles. WIND's data is essential for understanding the fundamental processes of the solar wind and its interaction with the Earth's magnetosphere.

\subsection{SOHO: Solar and Heliospheric Observatory}
SOHO, a joint project of ESA and NASA, was launched in 1995. It is designed to study the Sun from its core to the outer corona and the solar wind.

\subsubsection{LASCO: Large Angle and Spectrometric Coronagraph Observatory}
LASCO, one of the instruments on SOHO, observes the solar corona by creating an artificial eclipse. It is instrumental in detecting coronal mass ejections, which are significant drivers of space weather.

\section{Learnable Hermite Splines for continuization}

Implicit PINN research usually enriches datasets with high resolution simulation data. High resolution data plays an important role within the machine learning PDE solvers. Numerical integration methods' accuracy decays as the data resolution does so.
Given the unfulfilled knowledge about solar wind dynamics and low resolution nature of the data, this work utilizes polynomial interpolation methods with empirical backup \cite{windmodelling1} to convert the discrete problem into continuous.

\section{Coronagraph Analysis}

\section{Physics Informed Machine Learning Pipeline}

\subsection{Theoretical model}


\subsection{Induction Equation}
The induction equation describes the evolution of the magnetic field \( \mathbf{B} \) in the presence of the velocity field \( \mathbf{u} \):
\begin{equation}
\frac{\partial \mathbf{B}}{\partial t} = \nabla \times (\mathbf{u} \times \mathbf{B}) + \eta \nabla^2 \mathbf{B},
\end{equation}
where \( \eta \) is the magnetic diffusivity.

\subsection{Architecture}

\section{Methodology}

\subsection{Neural Architecture Search (NAS)}
The network is trained using a standard optimization algorithm such as Adam. During training, we compute the residuals for the momentum and induction equations, as well as the divergence-free constraint for the magnetic field, and include them in the loss function.

\subsection{Training Procedure}
The network is trained using a standard optimization algorithm such as Adam. During training, we compute the residuals for the momentum and induction equations, as well as the divergence-free constraint for the magnetic field, and include them in the loss function.

\section{Results}

\subsection{Coronagraph Model}

\subsection{L1 Lagrange Model}

\subsection{Overall}

\section{Discussion}

\subsection{Alternatives models}

\section{Conclusion}
This work demonstrated the effectiveness of Physics-Informed Neural Networks to model Solar Wind's behavior tied to diverse empirical and analytical formulations. By embedding the governing physical equations directly into the loss function, we can efficiently train a neural network to approximate the solution of this complex, non-linear system. Future work will focus on improving the kernel design and extending the approach to more sophisticated MHD scenarios, such as the interactions between Solar Wind and Sun's magnetosphere.

\bibliographystyle{plain}

\bibliography{references}
\end{document}
