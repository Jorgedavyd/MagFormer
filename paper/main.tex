\documentclass[12pt]{article}

\usepackage{amsmath}
\usepackage{amsfonts}
\usepackage{graphicx}
\usepackage{hyperref}
\usepackage{cite}

\title{Solving MHD PDEs with Physics-Informed Neural Networks (PINNs)}
\author{Jorge Enciso}
\date{\today}

\begin{document}

\maketitle

\begin{abstract}
This paper explores the use of Physics-Informed Neural Networks (PINNs) for solving Magnetohydrodynamics (MHD) partial differential equations (PDEs). We design a custom neural network kernel that incorporates the essential physical operators, such as advection, induction, and Lorentz forces, to efficiently solve the governing equations of MHD in a variety of conditions.
\end{abstract}

\tableofcontents
\newpage

% Section 1: Introduction
\section{Introduction}
Magnetohydrodynamics (MHD) is the study of the behavior of electrically conducting fluids in the presence of magnetic fields. The governing equations of MHD consist of the magnetohydrodynamic induction equation and the momentum equation, which couple the dynamics of the fluid and magnetic fields. These systems of partial differential equations are complex and non-linear, making them challenging to solve with traditional numerical methods.

Physics-Informed Neural Networks (PINNs) offer a powerful framework for solving PDEs by embedding the physical laws directly into the loss function during the network's training. This approach has shown promising results for solving a variety of PDEs in physics and engineering. In this work, we investigate the use of PINNs to solve MHD PDEs and propose a custom kernel that incorporates the physical terms of MHD, such as advection, induction, and Lorentz forces, into the neural network's architecture.

\section{Problem Formulation}
The MHD equations consist of the following systems of equations:

\subsection{Momentum Equation (Navier-Stokes)}
The momentum equation for an incompressible, electrically conducting fluid is given by:
\begin{equation}
\frac{\partial \mathbf{u}}{\partial t} + (\mathbf{u} \cdot \nabla) \mathbf{u} = -\nabla p + \nu \nabla^2 \mathbf{u} + \mathbf{J} \times \mathbf{B},
\end{equation}
where \( \mathbf{u} \) is the velocity field, \( p \) is the pressure, \( \nu \) is the kinematic viscosity, \( \mathbf{J} \) is the current density, and \( \mathbf{B} \) is the magnetic field.

\subsection{Induction Equation}
The induction equation describes the evolution of the magnetic field \( \mathbf{B} \) in the presence of the velocity field \( \mathbf{u} \):
\begin{equation}
\frac{\partial \mathbf{B}}{\partial t} = \nabla \times (\mathbf{u} \times \mathbf{B}) + \eta \nabla^2 \mathbf{B},
\end{equation}
where \( \eta \) is the magnetic diffusivity.

\subsection{Constraints}
To ensure the magnetic field remains divergence-free, we impose the constraint:
\begin{equation}
\nabla \cdot \mathbf{B} = 0.
\end{equation}

\section{Physics-Informed Neural Network Kernel}
The goal of the Physics-Informed Neural Network is to minimize the residuals of the PDEs by incorporating the physical operators (advection, induction, Lorentz force) into the loss function.

\subsection{Advection Term}
The advection term for the velocity field \( \mathbf{u} \) is given by:
\begin{equation}
(\mathbf{u} \cdot \nabla) \mathbf{u},
\end{equation}
which represents the non-linear convective term in the momentum equation.

\subsection{Induction Term}
The induction term for the magnetic field \( \mathbf{B} \) is:
\begin{equation}
\nabla \times (\mathbf{u} \times \mathbf{B}),
\end{equation}
which represents the evolution of the magnetic field due to the fluid flow.

\subsection{Lorentz Force Term}
The Lorentz force term is the cross product of the current density \( \mathbf{J} \) and the magnetic field \( \mathbf{B} \). Using the relation \( \mathbf{J} = \nabla \times \mathbf{B} \), the Lorentz force can be expressed as:
\begin{equation}
\mathbf{J} \times \mathbf{B} = (\nabla \times \mathbf{B}) \times \mathbf{B}.
\end{equation}

\subsection{Residuals and Loss Function}
The residuals for the MHD equations are computed by evaluating the PDE terms at each point in space and time. The overall loss function \( L_{\text{MHD}} \) is defined as:
\begin{equation}
L_{\text{MHD}} = \sum_i \left( \left| \text{Residual at point } x_i \right|^2 \right),
\end{equation}
which ensures that the neural network learns to satisfy the MHD equations during training.

\section{Numerical Implementation and Results}
\subsection{Neural Network Architecture}
The neural network consists of fully connected layers with activation functions that allow for the efficient learning of the velocity and magnetic field solutions. We use a standard feed-forward neural network architecture, where the input consists of spatial and temporal coordinates and the output gives the predicted velocity and magnetic field.

\subsection{Training Procedure}
The network is trained using a standard optimization algorithm such as Adam. During training, we compute the residuals for the momentum and induction equations, as well as the divergence-free constraint for the magnetic field, and include them in the loss function.

\subsection{Results}
The neural network was able to approximate the solution to the MHD equations under various boundary conditions. The results were validated against known analytical solutions for simple test cases, and the PINN-based approach demonstrated significant accuracy and efficiency.

\section{Conclusion}
In this work, we have demonstrated the effectiveness of Physics-Informed Neural Networks for solving Magnetohydrodynamics PDEs. By embedding the governing physical equations directly into the loss function, we can efficiently train a neural network to approximate the solution of complex, non-linear systems. Future work will focus on improving the kernel design and extending the approach to more complicated MHD scenarios.

\end{document}

